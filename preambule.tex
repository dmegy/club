\documentclass[11pt,a4paper]{book}
\usepackage[francais]{babel}
\usepackage[utf8]{inputenc}
\usepackage{amsmath,amssymb,amsthm}
\usepackage{mathrsfs,stmaryrd}
\usepackage{fancybox,mdframed,multicol,comment,datetime}
\usepackage{framed} % pour encadrer
\usepackage[dvipsnames]{xcolor}

\usepackage{fourier}
\usepackage{imakeidx}% imakeidx ne met pas les liens?
\usepackage[backref]{hyperref}
\hypersetup{
    colorlinks=true,       % false: boxed links; true: colored links
    linkcolor=[rgb]{0,0.2,0.6},          % color of internal links
    citecolor=[rgb]{0,0.2,0.6},        % color of links to bibliography
    filecolor=[rgb]{0,0.2,0.6},      % color of file links
    urlcolor=[rgb]{0.7,0.2,0.2}           % color of external links
}

\usepackage[all]{hypcap} % règle pb cible des liens hyperref
\usepackage{pgf,pgfmath,tikz}
\usetikzlibrary{arrows}
\usetikzlibrary[patterns]
\tikzset{every picture/.style={execute at begin picture={
   \shorthandoff{:;!?};}
}}

\usepackage[margin=2.5cm]{geometry}

\usepackage[normalem]{ulem} % pour souligner avec changements de ligne
\usepackage{esvect} % pour jolis vecteurs avec \vv

\everymath{\displaystyle} 



\theoremstyle{definition}
\newtheorem{theoreme}{Théorème}[chapter]
\newtheorem{definition}[theoreme]{Définition}
\newtheorem{lemme}[theoreme]{Lemme}
\newtheorem{proposition}[theoreme]{Proposition}
\newtheorem{corollaire}[theoreme]{Corollaire}
\newtheorem{remarque}[theoreme]{Remarque}
\newtheorem{ex}{Exercice}


\newcommand{\N}{\mathbb N}
\newcommand{\Z}{\mathbb Z}
\newcommand{\Q}{\mathbb Q}
\newcommand{\R}{\mathbb R}
\newcommand{\C}{\mathbb C}
\newcommand{\U}{\mathbb U}
\newcommand{\F}{\mathbb F}


%%%%%%%%%%%%%%%%%%%%%%%%%%%%%%%%%
%%%%%% MISE EN FORME CLUB %%%%%%%
%%%%%%%%%%%%%%%%%%%%%%%%%%%%%%%%%

%\pagestyle{empty}



% En-tête des feuilles :

\newcommand{\enTete}[1]{
\noindent \textbf{\textsf{\href{http://depmath-nancy.univ-lorraine.fr/club/}{Club Mathématique de Nancy} \hfill Institut Élie Cartan}}
\hrule
\begin{center}
{\Huge \textbf{#1}}
\end{center}
\hrule
\vspace{1em}
}


%----- Structure des exercices ------

\newenvironment{prerequis}{(\bfseries Prérequis:}{)\newline}

