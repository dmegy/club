\begin{Hint}{4}
Pour commencer, utiliser le principe des tiroirs.
\end{Hint}
\begin{Hint}{5}
\'Etudier des exemples en posant la division à la main. Traiter par exemple le cas de $\frac{13}{7}$.
\end{Hint}
\begin{Hint}{7}
On a même envie de dire plus, à savoir que l'inégalité est même en générale stricte, et qu'il y a égalité que si les quatre points sont les sommets d'un carré et dans ce cas-là toutes les distances valent $\sqrt2$. Cette remarque fait penser à une inégalité sur la moyenne, comme dans la preuve du principe des tiroirs.
\end{Hint}
\begin{Hint}{8}
Ordonner les entiers.
\end{Hint}
\begin{Hint}{9}
Si on veut appliquer le principe des tiroirs, combien de tiroirs faut-il ?
\end{Hint}
\begin{Hint}{10}
Ici, les tiroirs ne sont pas des parties du plan : ce sont des tiroirs \og abstraits\fg. Pour faire marcher le raisonnement, la propriété d'être dans le même tiroir doit impliquer que le milieu du segment a des coordonnées entières.
\end{Hint}
\begin{Hint}{11}
Combien de tiroirs faut-il ? Quelle propriété doivent-ils vérifier ?
\end{Hint}
\begin{Hint}{14}
En termes de divisibilité, qu'est-ce qu'un nombre composé des deux mêmes chiffres ?
\end{Hint}
\begin{Hint}{15}
De façon très générale, si un ensemble a $n$ éléments, combien a-t-il de parties ?
\end{Hint}
\begin{Hint}{19}
On peut constater que $101=\left(11-1\right)\left(11-1\right)+1$.
\end{Hint}
\begin{Hint}{21}
 Méthodologie : de quels théorèmes dispose-t-on ? Lesquels concernent le parallélisme ? Pour l'aire, considérer l'aire du complémentaire de $IJKL$ par exemple, ou bien utiliser les diagonales de $ABCD$.
\end{Hint}
\begin{Hint}{22}
 %Sans utiliser le "théorème des milieux", on peut compléter le trapèze rectangle en un rectangle grâce à la symétrie de centre $I$.
Les diagonales d'un rectangle sont égales et se coupent en leur milieu. % Y a-t-il plus simple ?
\end{Hint}
\begin{Hint}{24}
 Méthodologie : essayer avec plusieurs points $M$. Que remarque-t-on ?


% autre solution, voir vieux TD, rotations pour se ramener à un seul côté ?
\end{Hint}
\begin{Hint}{25}
\'Ecrire les distances aux sommets en fonction des angles du triangle.
\end{Hint}
\begin{Hint}{26}
Triangles isocèles.
\end{Hint}
\begin{Hint}{27}
Où se trouve le point $M$ sur le segment $[AC]$ ?
\end{Hint}
\begin{Hint}{28}
Faire tourner le carré circonscrit par rapport au carré inscrit.
\end{Hint}
\begin{Hint}{29}
 Considérer l'application du segment $[AB]$ dans lui-même qui a un point $D$ sur le segment associe $G$ comme construit dans l'énoncé. Que dire si $D$ est une des extrémités du segment ? Que peut-on dire de cette application ? % affine, et involutive car échange deux points; en fait c'est le symétrique sur le segment
\end{Hint}
\begin{Hint}{30}
Les entiers semblent divisibles par trois dans le premier cas, par cinq dans le second cas.
\end{Hint}
\begin{Hint}{32}
Les sommes semblent toujours donner des carrés.
\end{Hint}
\begin{Hint}{33}
Montrer qu'il existe forcément un dépôt d'essence contenant assez d'essence pour rejoindre le suivant.
\end{Hint}
\begin{Hint}{34}
Penser aussi à des échiquiers de taille différente.
\end{Hint}
\begin{Hint}{38}
\begin{enumerate}
\item
\item Penser à la récurrence
\end{enumerate}
\end{Hint}
\begin{Hint}{39}
On peut tester à la main pour des petites valeurs de $n$.
\end{Hint}
\begin{Hint}{42}
Récurrence forte.
\end{Hint}
\begin{Hint}{44}
Faire intervenir la parité des entiers.
\end{Hint}
\begin{Hint}{47}
Montrer qu'il existe au moins une \emph{oreille}, c'est-à-dire un triangle avec deux arêtes appartenant à la frontière du polygone, et la troisième située à l'intérieur du polygone.

Plus précisément, montrer qu'il existe toujours deux oreilles  qui ne se chevauchent pas.
\end{Hint}
\begin{Hint}{48}
Pour la dernière question, démontrer que pour tout $n\geq 2$, on a :\[F_n^2 = F_{n-1}\,F_{n+1} + (-1)^{n+1}.\]
\end{Hint}
\begin{Hint}{50}
Pour l'existence, procéder par récurrence forte.

Pour l'unicité, démontrer préalablement le lemme suivant : la somme de tout ensemble de nombres de Fibonacci distincts et non consécutifs, dont le plus grand élément est $F_j$, est strictement inférieure à $F_{j + 1}$.
\end{Hint}
\begin{Hint}{51}
Procéder par récurrence forte.
\end{Hint}
\begin{Hint}{52}
Penser à initialiser sur les deux premiers rangs.
\end{Hint}
\begin{Hint}{53}
 % Regine 21/03/18
On rappelle que $v=d/t$. Essayer d'exprimer la distance totale parcourue de deux manières.
\end{Hint}
\begin{Hint}{54}
 % Regine 21/03/18
Augmenter de $x\%$ est la même chose qu'être multiplié par $\displaystyle \left( 1+\frac{x}{100} \right)$. \\
Essayer d'écrire de deux manières la population après deux siècles.
\end{Hint}
\begin{Hint}{55}
 % Regine 21/03/18
Essayer d'écrire de deux manières l'énergie totale dissipée.
\end{Hint}
\begin{Hint}{56}
 % Regine 21/03/18
Essayer d'écrire de deux manières la durée totale du trajet.
\end{Hint}
\begin{Hint}{57}
Introduire les longueurs $a$ et $b$ des côtés du rectangle. Essayer de traduire l'inégalité arithmético-géométrique en une inégalité faisant intervenir l'aire et le périmètre du rectangle.
\end{Hint}
\begin{Hint}{58}
Essayer de traduire l'inégalité arithmético-géométrique en une inégalité faisant intervenir le produit et la somme des deux nombres.
\end{Hint}
\begin{Hint}{59}
Introduire $x$, la longueur commune des côtés $AB$ et $CD$.
\end{Hint}
\begin{Hint}{60}
Introduire les longueurs $a$, $b$, $c$ des côtés, exprimer l'aire des faces à l'aide de ces variables et appliquer l'inégalité arithmético-géométrique à cette expression.
\end{Hint}
\begin{Hint}{61}
On a déjà fait quelque chose d'analogue dans l'exercice \ref{EXO:sp1}.
\end{Hint}
\begin{Hint}{62}
Exprimer les deux quantités à comparer en fonction de $a,b,c$ et l'inégalité arithmético-géométrique devrait pointer le bout de son nez...
\end{Hint}
\begin{Hint}{63}
 Rien !
 
\end{Hint}
\begin{Hint}{64}
Appliquer l'inégalité arithmético-géométrique à chacun des facteurs du produit.
\end{Hint}
\begin{Hint}{65}
Commencer par développer le produit et l'exprimer en fonction de $ab$.
\end{Hint}
\begin{Hint}{66}
Commencer par tout développer et simplifier.
\end{Hint}
\begin{Hint}{67}
Il pourrait être intéressant de minorer $2^x+\frac{1}{2^x}$.
\end{Hint}
\begin{Hint}{71}
Utiliser les exercices précédents.
\end{Hint}
\begin{Hint}{74}
 On peut écrire l'aire à l'aide d'un sinus, puis à l'aide d'un cosinus, puis utiliser le théorème d'Al Kashi.
\end{Hint}
\begin{Hint}{75}
Utiliser la formule de Héron et l'inégalité arithmético-géométrique.
\end{Hint}
\begin{Hint}{78}
Essayer de calculer cette somme en utilisant les arêtes.
\end{Hint}
\begin{Hint}{79}
Utiliser l'exercice \ref{EXO:grsomdeg}.
\end{Hint}
\begin{Hint}{80}
Utiliser l'exercice \ref{EXO:grsomdeg}.
\end{Hint}
\begin{Hint}{81}
Si tous les sommets ont des degrés différents, quels sont nécessairement ces degrés ?
\end{Hint}
\begin{Hint}{82}
Raisonner par l'absurde.
\end{Hint}
\begin{Hint}{83}
Montrer qu'en partant d'un point arbitraire, on peut construire un chemin de sorte que n'importe quels $k+1$ points successifs du chemins sont deux à deux distincts.
%Construire un tel cycle en partant d'un point arbitraire.
\end{Hint}
\begin{Hint}{84}
Remarquer que quand on ajoute une arête à un arbre maximal, on crée un cycle, et que quand on ôte une arête d'un arbre, l'ensemble de ses sommets n'est plus connexe.
\end{Hint}
\begin{Hint}{85}
Considérer un chemin \emph{élémentaire} (ne passant pas plus d'une fois par chaque sommet) et maximal pour cette propriété.
\end{Hint}
\begin{Hint}{86}
Utiliser l'exercice \ref{EXO:arbrefeuille} pour trouver un sommet de degré $1$, et faire une récurrence.
\end{Hint}
\begin{Hint}{87}
Si, à un sous-graphe connexe, on ajoute une arête tout en restant connexe, combien de nouveaux sommets a-t-on ajoutés ?
\end{Hint}
\begin{Hint}{89}
Un chemin qui joint deux points est toujours plus long que le plus court chemin entre ces deux points...
\end{Hint}
\begin{Hint}{90}
Si, dans un graphe dont les sommets sont tous de degré pair, on retire les arêtes d'un cycle, qu'obtient-on  ?
\end{Hint}
\begin{Hint}{91}
S'il y a deux sommets de degré impair, ajouter une arête entre ces deux sommets.
\end{Hint}
\begin{Hint}{92}
\begin{enumerate}
\item Procéder par récurrence sur le nombre d'arêtes du graphe.
\item On pourra considérer le cas $k=1$, et  interpréter dans ce cas la quantité $a-s+1$.% C'est le nombre de cycles élémentaires : un arbre couvrant possède $s-1$ arêtes : donc $a-(s-1)$ est le nombre d'arêtes ajoutées à un arbre couvrant.
\end{enumerate}
\end{Hint}
\begin{Hint}{93}
Que vaut la quantité $0+1+2+3+...+l$ ?
\end{Hint}
\begin{Hint}{99}
Il suffit de résoudre le problème sans les doubles, que l'on peut intercaler par la suite.

%Sans les doubles, on se retrouve avec un graphe complet, et un graphe complet est hamiltonien.
% ATTENTION, ERREUR ? Voir remarque de Tom et Clémence, refaire.
\end{Hint}
\begin{Hint}{101}
Il y a deux tels octogones. En notant $O$ le centre d'un tel octogone, on doit avoir $\widehat{AOB}=\pm \pi/4$.
\end{Hint}
\begin{Hint}{105}
Montrer que $BIC$ est isocèle en $I$.
\end{Hint}
\begin{Hint}{106}
Condition de cocyclicité.
\end{Hint}
\begin{Hint}{109}
Utiliser un des exercices précédents.% celui sur l'intersection d'une bissectrice et d'une médiatrice, par exemple. Mais là c'est un cas particulier.
\end{Hint}
\begin{Hint}{110}
Triangles isocèles et rectangles
\end{Hint}
\begin{Hint}{111}
Introduire la tangente commune $\mathcal T$ aux deux cercles.
%Utiliser le cas limite du théorème de l'angle au centre.
\end{Hint}
\begin{Hint}{112}
Utiliser les angles droits pour montrer que des points sont cocycliques, puis utiliser le théorème de l'angle inscrit.
\end{Hint}
\begin{Hint}{113}
Sans le théorème de Ptolémée, on peut considérer le point $N \in [AM]$ tel que $\widehat{BNM}=\pi/3$.
\end{Hint}
\begin{Hint}{115}
Utiliser les différentes caractérisations des triangles isocèles.
\end{Hint}
\begin{Hint}{116}
Où se trouve le centre du triangle circonscrit d'un triangle rectangle ?
\end{Hint}
\begin{Hint}{118}
On pourra procéder par récurrence sur le nombre de sommets du graphe, ou bien construire une représentation planaire explicite en donnant les coordonnées des points et les chemins pour les relier.
\end{Hint}
\begin{Hint}{119}
Penser à la somme des degrés des sommets.
\end{Hint}
\begin{Hint}{121}
Non, c'est impossible. Le graphe à considérer est $K_5$, le graphe complet sur cinq sommets, donc voici une représentation (non planaire).
\begin{center}
\definecolor{uuuuuu}{rgb}{0.26666666666666666,0.26666666666666666,0.26666666666666666}
\definecolor{qqqqff}{rgb}{0.3333333333333333,0.3333333333333333,0.3333333333333333}
\begin{tikzpicture}[line cap=round,line join=round,>=triangle 45,x=.6cm,y=.6cm]
\clip(-2.14,-0.98) rectangle (2.7,3.94);
\draw (-1.46,0.04)-- (-1.6566884575995895,2.457542895306533);
\draw (-1.6566884575995895,2.457542895306533)-- (0.5817513904090712,3.3916665738667984);
\draw (0.5817513904090712,3.3916665738667984)-- (2.1618717558501617,1.5514438616065909);
\draw (0.9,-0.52)-- (2.1618717558501617,1.5514438616065909);
\draw (0.9,-0.52)-- (-1.46,0.04);
\draw (-1.46,0.04)-- (0.5817513904090712,3.3916665738667984);
\draw (-1.46,0.04)-- (2.1618717558501617,1.5514438616065909);
\draw (0.9,-0.52)-- (-1.6566884575995895,2.457542895306533);
\draw (2.1618717558501617,1.5514438616065909)-- (-1.6566884575995895,2.457542895306533);
\draw (0.5817513904090712,3.3916665738667984)-- (0.9,-0.52);
\begin{scriptsize}
\draw [fill=qqqqff] (-1.46,0.04) circle (2.5pt);
%\draw[color=qqqqff] (-1.98,0.41) node {$A$};
\draw [fill=qqqqff] (0.9,-0.52) circle (2.5pt);
%\draw[color=qqqqff] (1.4,-0.21) node {$B$};
\draw [fill=uuuuuu] (2.1618717558501617,1.5514438616065909) circle (2.5pt);
%\draw[color=uuuuuu] (2.3,1.93) node {$C$};
\draw [fill=uuuuuu] (0.5817513904090712,3.3916665738667984) circle (2.5pt);
%\draw[color=uuuuuu] (0.72,3.77) node {$D$};
\draw [fill=uuuuuu] (-1.6566884575995895,2.457542895306533) circle (2.5pt);
%\draw[color=uuuuuu] (-1.52,2.83) node {$E$};
\end{scriptsize}
\end{tikzpicture}
\end{center}
Il s'agit de montrer que ce graphe n'est pas planaire.
\end{Hint}
\begin{Hint}{124}
Montrer qu'il existe au moins une face de degré $\leq 5$, et qu'il existe au moins un sommet de degré $\geq 3$.
\end{Hint}
\begin{Hint}{125}
1)Procéder par récurrence et utiliser un des résultats précédents pour l'hérédité.

2) Procéder par récurrence comme dans la question précédente. Par contre, au moment de rajouter le sommet supplémentaire, il faut modifier le coloriage précédent d'une certaine façon pour justifier que cinq couleurs suffisent toujours.
\end{Hint}
\begin{Hint}{126}
L'idée est qu'un graphe complet a \og beaucoup\fg{} d'arêtes, alors que des graphes planaires n'ont que \og peu\fg{} d'arêtes.
\end{Hint}
\begin{Hint}{128}
 %Régine 21/03/18
Les cartes d'une main \emph{ne sont pas ordonnées}.\\
Pour la question 6: dans un jeu de 48 cartes (4 \emph{couleurs} et de 12 \emph{valeurs} possibles), combien de mains de deux cartes de valeurs différentes peut-on faire ? Attention à ne pas compter une même main plusieurs fois.
\end{Hint}
\begin{Hint}{129}
 %Régine 21/03/18
On peut essayer de représenter les solutions par un arbre.  On part de la racine, et on se demande combien de possibilités on a pour la place la plus à gauche sur l'étagère: on a $n$ choix possibles, donc on fait partir $n$ branches de la racine. On va ensuite à l'extrémité de l'une de ces branches: combien de choix possibles pour la seconde place la plus à gauche ? $n-1$, donc on met $n-1$ nouvelles branches...
\end{Hint}
\begin{Hint}{130}
 %Régine 21/03/18
De combien de segments de rue \og vers l'est \fg{} et de combien de segments \og vers le nord \fg{} un chemin de longueur $9$ allant de $(0,0)$ à $(4,5)$ est-il composé ?
\end{Hint}
\begin{Hint}{131}
 %Régine 21/03/18
On peut essayer de représenter le problème par un chemin comme dans l'exercice précédent, en mettant en abscisse les employés numérotés de $1$ à $k$, et en ordonnée le nombre de cadeaux distribués.
\end{Hint}
\begin{Hint}{132}
 %Régine 21/03/18
Partitionner les partages en $k$ parts de l'entier $n$ entre ceux qui contiennent des parts égales à $1$ et ceux qui n'en contiennent pas.
Faire des diagrammes de Ferrers.
\end{Hint}
\begin{Hint}{133}
 %Régine 21/03/18
Combien peut-on écrire de mots de $n$ lettres avec l'alphabet $\{a,b\}$ ? Combien peut-on écrire de mots palindromes de $n$ lettres avec l'alphabet $\{a,b\}$ ? on pourra traiter séparément le cas $n$ pair et $n$ impair.
\end{Hint}
\begin{Hint}{134}
 %Régine 21/03/18
Comme on peut faire tourner la table, on peut dire qu'on place Alice à la place $1$, et numéroter les places restantes de $2$ à $n$ dans le sens des aiguilles d'une montre.
\end{Hint}
\begin{Hint}{135}
 %Régine 21/03/18
Trier les partitions de $\{1, \dots, n+1\}$ en fonction du nombre d'éléments qui ne sont pas dans la même partie que $n+1$.
\end{Hint}
\begin{Hint}{136}
 %Régine 21/03/18
Combien d'anagrammes peut-on former à partir de $9$ lettres deux à deux distinctes ? Quelle est le problème quand une même lettre apparaît plusieurs fois ?
\end{Hint}
\begin{Hint}{144}
Il s'agit bien d'une heure, et pas d'une heure et demie. Ce n'est donc pas exactement le même exercice que les précédents.
\end{Hint}
\begin{Hint}{160}
Il y a quatre couples de solutions.
\end{Hint}
