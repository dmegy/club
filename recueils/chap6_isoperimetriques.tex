

%%%%%%%%%%%%%%%%%%%%%%%%%%%%%
\chapter{Problèmes isopérimétriques}
%%%%%%%%%%%%%%%%%%%%%%%%%%%%%


Un problème isopérimétrique est un problème de maximisation de l'aire d'une figure, à périmètre fixé (\emph{iso} = même). Dans cette feuille, on s'intéresse aux polygones.

Dans toute la suite, un polygone est dit \textbf{optimal} si de tous les polygones ayant le même périmètre et le même nombre de côtés, il a la plus grande aire. Le fait de savoir s'il existe des polygones optimaux est difficile. Dans cette feuille, on démontre entre autres:
\begin{enumerate}
\item Pour les polygones à trois côtés : il existe des $3$-gones (triangles) optimaux : ce sont les triangles équilatéraux.
\item Pour les polygones à quatre côtés : il existe des $4$-gones (quadrilatères) optimaux : ce sont les carrés.
\item Pour un nombre $n\geq 5$ de côtés, le problème est plus difficile. On démontre que si un polygone n'est \emph{pas} convexe et régulier, alors il n'est \emph{pas} optimal, autrement dit on peut trouver un autre polygone ayant même nombre de côtés, même périmètre, et aire plus grande. 
\end{enumerate}

Remarque : le résultat général est que les $n$-gones optimaux existent et sont les $n$-gones réguliers convexes, mais les résultats de cette feuille ne suffisent pas pour établir ce théorème, car ils ne montrent pas l'\emph{existence} d'une solution, mais juste que si elle existe, alors c'est celle-là.


Certains exercices utilisent l'inégalité arithmético-géométrique, à deux ou trois variables. Les exercices dépendent parfois les uns des autres (dans l'ordre de la feuille)

%%%%%%%%%%%%%%%%%
\begin{exo}[Parallélogrammes optimaux]
\label{parallelogramme}
Soit $p$ un nombre réel strictement positif. 
\begin{enumerate}
\item Déterminer le ou les rectangles de périmètre égal à $p$ dont l'aire est maximale pour ce périmètre.
\item Même question avec les parallélogrammes de périmètre $p$.
\end{enumerate}
\begin{sol}
\begin{enumerate}
\item 
Soient $a$ et $b$ les mesures des côtés du rectangle. L'aire du rectangle vaut donc: 
\[ \mathcal A=ab.\]
 
D'autre part, en  calculant le périmètre $p$  en fonction de $a$ et $b$ on obtient la contrainte :
\[ 2(a+b)=p.\]

Il s'agit donc de déterminer $a$ et $b$ tels que $a+b=p/2$, de telle façon à maximiser la quantité $ab$. Or, l'inégalité arithmético-géométrique donne justement :
\[ \sqrt ab \leq \frac{a+b}{2} \]
avec égalité si et seulement si $a=b$, autrement dit, en élevant au carré et en écrivant le résultat en fonction de $p$ et de $\mathcal A$:
\[ \mathcal A \leq \frac{p^2}{16},\]
avec égalité ssi $a=b$.

Ceci montre qu'à périmètre fixé, l'aire est toujours inférieure à $p^2/16$ et que cette aire maximale est atteinte exactement lorsque les deux côtés du rectangle sont égaux, c'est-à-dire lorsque le rectangle est un carré. (La quantité $p^2/16$ est bien l'aire du carré de côté $a=b=p/4$.)



\item Pour un parallélogramme, on note $x=AB$ et $y=BC$ les longueurs de deux côtés consécutifs, et $\theta = \widehat{ABC}$. Alors l'aire du parallélogramme est $\mathcal A = xy\sin\theta$.  Si $x$ et $y$ sont variables sous la contrainte que $x+y = p/2$ est constant, alors on peut écrire la majoration
\[\mathcal A = xy\sin\theta \leq xy,\]
car un sinus est toujours inférieur ou égal à $1$. (Rappelons que c'est une conséquence du théorème de Pythagore : dans un triangle rectangle, l'hypoténuse est plus grande que les deux autres côtés. Ceci entraîne qu'un cosinus et un sinus sont toujours inférieurs ou égaux à $1$.)

Cette inégalité est une égalité si et seulement si le sinus vaut $1$, c'est-à-dire si et seulement si l'angle $\widehat{ABC}$ est droit, autrement dit si et seulement si le parallélogramme est un rectangle.

On en déduit qu'à longueurs des côtés $x$ et $y$ fixées, le parallélogramme ayant la plus grande aire est le rectangle de côtés $x$ et $y$.

On peut alors appliquer la première question : à périmètre $p = 2x+2y$ fixé, le rectangle ayant la plus grande aire est le carré de côté $p/4$.
\end{enumerate}
\end{sol}
\end{exo}



%\begin{center}
%$ \star \star \star$ Triangles $\star \star \star$
%\end{center}

%%%%%%%%%%%%%%%%%
\begin{exo}[Périmètre minimal à aire fixée]
Soient $A$ et $B$ deux points, et $\mathcal A$ un réel. Parmi tous les triangles $ABC$ d'aire $\mathcal A$, déterminer celui ou ceux dont le périmètre est minimal.
\begin{sol}
Il y a deux triangles vérifiant ces conditions : ce sont les deux triangles isocèles de base $AB$ et d'aire $\mathcal A$. Pour le démontrer, considérons les deux droites parallèles à $(AB)$ et à une distance $h =\frac{\mathcal A}{ AB}$ de la droite $(AB)$. Un triangle $ABC$ a une aire égale à $\mathcal A$ si et seulement si le point $C$ appartient à un de ces deux droites (suivant s'il est direct ou indirect).

Pour simplifier, supposons que l'on cherche uniquement les triangles $ABC$ directs, et appelons $\mathcal D$ la droite formée des points $C$ tels que $ABC$ soit direct et d'aire $\mathcal A$.

Il s'agit donc de trouver un tel point $C$ pour que le périmètre de $ABC$ soit minimal. Comme la distance $AB$ est fixe, il s'agit donc de minimiser $AC+CB$. On peut appliquer le résultat de l'exercice \ref{riviere}, d'après lequel le point $C$ qui convient est l'intersection de $\mathcal D$ et du segment joignant $A$ à l'image de $B$ par symétrie axiale d'axe $\mathcal D$. Ceci produit un triangle isocèle (utilisation d'angles opposés, ou alors alternes-internes).
\end{sol}
\end{exo}

%%%%%%%%%%%%%%%%%
\begin{exo}[Aire maximale à périmètre fixé]
\label{base_fixee}
Soient $A$ et $B$ deux points. Parmi tous les triangles de périmètre $p$ fixé, trouver celui ou ceux d'aire maximale.
\begin{sol}


Ce sont les deux triangles isocèles de base $[AB]$. Le problème est dual du précédent.

Rédaction possible : si $ABC$ est un triangle non isocèle en $C$, on peut d'après l'exercice précédent trouver un triangle de même base $[AB]$ et de même aire, qui est isocèle, et de périmètre inférieur. Ensuite, on augmente progressivement la hauteur de ce triangle jusqu'à avoir à nouveau un périmètre égal à $p$, mais ceci a augmenté l'aire. Ceci montre qu'on triangle $ABC$ de périmètre $p$ a une aire toujours inférieure à un triangle isocèle de base $[AB]$ de périmètre $p$.



Autre solution, de plus haut niveau : fixer le périmètre revient à fixer la longueur $AC+CB$. Le point $C$ décrit donc une ellipse.

D'autre part,  par la formule $\mathcal A = AB \cdot h$ avec $h$ la longueur de la hauteur issue de $C$, le  ou les triangles d'aire maximale sont ceux pour lesquels $C$ est le plus éloigné de la droite $(AB)$.

On voit alors que l'aire maximale est atteinte lorsque $C$ est sur a médiatrice de $[AB]$.
\end{sol}
\end{exo}

%%%%%%%%%%%%%%%%%
\begin{exo}[Quadrilatères optimaux]
\label{quadrilateres}
Soit $ABCD$ un quadrilatère non croisé.
\begin{enumerate}
\item Montrer que s'il n'est pas convexe, il existe un quadrilatère convexe de même périmètre et d'aire supérieure. Dans la suite, on supposera le quadrilatère convexe.
\item Montrer qu'il existe un losange de même périmètre et de plus grande aire.
\item En déduire que les quadrilatères non croisés de périmètre $p$ et d'aire maximale sont ceux qui sont carrés (donc de coté $p/4$ et d'aire $p^2/16$).
\end{enumerate}
\begin{hint}
Utiliser les exercices précédents.
\end{hint}
\begin{sol}
\begin{enumerate}
\item On peut supposer que $[AC]$ est à l'extérieur de $ABCD$. Soit $B'$ le symétrique de $B$ par rapport à $(AC)$. Alors $AB'CD$ est convexe, de même périmètre (une symétrie axiale conserve les distances), et d'aire supérieure, car on a ajouté l'aire de $ABCB'$ à l'aire initiale.
\item Si $ABCD$ n'est pas un losange, il existe forcément deux coté de longueur différente. On peut supposer que c'est $[AB]$ et $[BC]$.

Soit $B'$ le point tel que $AB'C$ soit isocèle en $B'$, du même coté de $[AC]$ que $ABC$, et de même périmètre que $ABC$. Par l'exercice \ref{base_fixee}, l'aire de $AB'C$ est supérieure à celle de $ABC$. De la même façon, on construit un point $D'$ tel que $CD'A$ soit isocèle en $D'$, de même périmètre que $CDA$, et son aire est alors supérieure.

Ceci montre que le quadrilatère $AB'CD'$, qui vérifie $AB'=B'C$ et $CD'=D'A$ et donc est un cerf-volant, a le même périmètre que $ABCD$ mais une aire supérieure.

Si $B'C=CD'$, alors $AB'CD'$ est un losange. Sinon, on répète l'opération en remplaçant $A$ et $C$ par $A'$ et $C'$ vérifiant $B'C'=C'D'$ et $B'A'=A'D'$, le périmètre étant toujours le même.

Finalement, on obtient bien un losange de même périmètre et d'aire supérieure.

\item Par l'exercice \ref{parallelogramme}, on peut alors trouver un carré de même périmètre que le losange et d'aire encore supérieure.
\end{enumerate}
\end{sol}
\end{exo}

%%%%%%%%%%%%%%%%%
\begin{exo}[Deux côtés fixés]
\label{deux_cotes_fixes}
Soient $a$ et $b$ deux réels strictement positifs. Déterminer les triangles dont deux côtés sont de longueur $a$ et $b$, et qui sont d'aire maximale.
\begin{sol}
Ce sont les triangles rectangles. Avec de la trigonométrie on peut écrire l'aire $\frac12 ab\sin(\gamma)$.
\end{sol}
\end{exo}

%%%%%%%%%%%%%%%%%
\begin{exo}[Loi des cosinus]
\label{Al_Kashi}
Soit $ABC$ un triangle, et $a$, $b$ et $c$ les longueurs de ses côtés. Montrer la \og loi des cosinus\footnote{Dans les manuels scolaires français des années 1990, ce résultat est appelé \og formule d'Al Kashi\fg{} (mais il semble ne plus être explicitement au programme).}\fg: 
\[ a^2=b^2+c^2-2bc\cos(\widehat{BAC}).\]


\begin{sol}
Il existe de très nombreuses preuves. La plus courte est celle utilisant l'outil le plus évolué disponible en géométrie plane au lycée, à savoir le produit scalaire:
\[ a^2 = BC^2=||\overrightarrow{CB}||^2 = ||\overrightarrow{AB} - \overrightarrow{AC}  ||^2 =||\overrightarrow{AB}||^2-2\langle \overrightarrow{AB},\overrightarrow{AC}\rangle+||\overrightarrow{AC}||^2= b^2+c^2 - 2bc \cos(\alpha).\]

Sans produit scalaire, en prenant une hauteur et en utilisant Pythagore:

\[  a^2 = h^2+c_2^2 = (b^2-c_1^2) + (c-c_1)^2 = b^2-c_1^2+c^2-2cc_1 +c_1^2=b^2+c^2 - 2bc\cos(\alpha).\]

Pour d'autres preuves, voir \url{https://fr.wikipedia.org/wiki/Loi_des_cosinus}
\end{sol}
\end{exo}


%%%%%%%%%%%%%%%%%
\begin{exo}[Formule de Héron]
\label{formule_Heron}
Soit $ABC$ un triangle, $a$ $b$ et $c$ les longueurs de ses cotés, $s=(a+b+c)/2$ son demi-périmètre et $\mathcal A$ son aire. Montrer que
\[\mathcal A = \sqrt{s(s-a)(s-b)(s-c)}.\]

\begin{hint} On peut écrire l'aire à l'aide d'un sinus, puis à l'aide d'un cosinus, puis utiliser le théorème d'Al Kashi.
\end{hint}

\begin{sol}
\begin{align*}
|\sin(\alpha)| &= \sqrt{1-\cos^2(\alpha)} \\
&= \sqrt{1-\left(\frac{a^2-b^2-c^2}{2bc}\right)^2}
\end{align*}
en utilisant la formule d'Al-Kashi.


D'où:
\begin{align*}
\frac{bc|sin\alpha|}{2} &= \frac{1}{4}\sqrt{(4b^2c^2+a^2-b^2-c^2)(4b^2c^2-a^2-b^2+c^2)} \\
&= \sqrt{\frac{(a+b+c)(b+c-a)(a-b+c)(a+b-c)}{16}}\\
&= \sqrt{p(p-a)(p-b)(p-c)}
\end{align*}
\end{sol}
\end{exo}

%%%%%%%%%%%%%%%%%
\begin{exo}[Triangles optimaux]
Soit $p$ un réel strictement positif. Montrer que les triangles de périmètre $p$ et d'aire maximale pour ce périmètre sont les triangles équilatéraux de périmètre $p$.
\begin{hint}
Utiliser la formule de Héron et l'inégalité arithmético-géométrique.
\end{hint}
\begin{sol}
D'après la formule de Héron, on a 
\[ \mathcal A = \sqrt{s(s-a)(s-b)(s-c)}.\]
Essayons donc de majorer l'aire $\mathcal A$ par l'air qu'aurait un triangle équilatéral de périmètre $p=2s$, c'est-à-dire par $\frac12 \cdot \frac{p}{3}\cdot\frac{p}{3}\cdot \frac{\sqrt 3}{2} = \frac{s^2}{3\sqrt 3}$.

Par inégalité arithmético-géométrique à trois variables appliquée à $s-a$, $s-b$ et $s-c$, on a:
\begin{align*}
(s-a)(s-b)(s-c)
&\leq (\frac{3s-a-b-c}{3})^3\\
&\leq (\frac{3s-2s}{3})^3\\
&\leq (\frac{s}{3})^3.
\end{align*}
avec égalité ssi $a=b=c$.

On a donc
\begin{align*}
\sqrt{s(s-a)(s-b)(s-c)} 
&\leq \sqrt{s^4/27} \\
&\leq  s^2/\sqrt{27}\\
&\leq \frac{s^2}{3\sqrt 3}.
\end{align*}

\end{sol}
\end{exo}






%%%%%%%%%%%%%%%%%
\begin{exo}[Non équilatéral $\implies$ non optimal]
Soit $p>0$ un réel, et $\mathcal P = A_1A_2...A_n$ un polygone à $n$ côtés de périmètre $p$. 
\begin{enumerate}
\item S'il n'est pas convexe, montrer qu'il n'est pas optimal.
\item S'il n'est pas équilatéral, montrer qu'il n'est pas optimal.
\end{enumerate}
\begin{sol}
\begin{enumerate}
\item S'il y a un creux, une réflexion axiale ou une symétrie centrale permettent de construire un autre polygone.
\item Si deux cotés $AB$ et $BC$ ne sont pas égaux, on utilise l'exercice \ref{base_fixee} appliqué au triangle $ABC$ pour construire un point $B'$ tel que $AB+BC = AB'+B'C$, et $AB'C$ isocèle.
\end{enumerate}
\end{sol}
\end{exo}



%%%%%%%%%%%%%%%%%
\begin{exo}[Non régulier $\implies$ non optimal]
Soit $p>0$ un réel et $\mathcal P = A_1A_2...A_{2n}$ un polygone de périmètre $p$, équilatéral et avec un nombre pair $2n$ de côtés. Montrer que si ses sommets ne sont pas tous sur un même cercle, alors il n'est pas optimal.
\end{exo}



%%%%%%%%%%%%%%%%%%
%\begin{exo}[Pour finir] 
%Trouver deux polygones ayant même aire, même périmètre, mais un nombre de côtés différents. (Si possible, trouver plusieurs exemples.)
%\end{exo}



%\chapter*{La véritable inégalité isopérimétrique et des applications}

%Problème de Didon 

%Problème de Didon avec un "cap", saillant ou rentrant.

